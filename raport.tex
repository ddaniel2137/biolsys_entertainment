\documentclass{article}
\usepackage[top=1in, bottom=1in, left=1.25in, right=1.25in]{geometry}
\usepackage[utf8]{inputenc}
\usepackage{amsmath}
\usepackage{graphicx}
\usepackage{float}

\title{Evolutionary Model \\
    based on Fisher's Geometric Model}
\author{Daniel Duda}
\date{\today}

\begin{document}

\maketitle

\section{Introduction}

The goal of this project was to create an evolutionary model based on Fisher's Geometric Model. Subsequently, we have
performed a series of simulations to study the effects of mutation probability and effect on the adaptation of the population.
Additionally, our specific aim was to 


\section{Methodology}

The model is

\begin{itemize}
    \item[\textbf{1.}] Individual:
        \begin{itemize}
            \item Each individual is represented by a numpy array of shape \verb|(n_traits,)| where \verb|n_traits|
            is the number of genetic traits and each trait is a float number in the range $[-1, 1]$
            \item The number of traits can be changed easily in the code using the \verb|n_traits| parameter
            \item Using a numpy array instead of a list allows for vectorized operations and thus is much
            more efficient computationally
            \item There is no sex trait. The model assumes that all individuals are hermaphrodites, but only
            one individual can reproduce in each generation
        \end{itemize}
    \item[\textbf{2.}] Population:
        \begin{itemize}
            \item Initially population is represented by a list of individuals of length \verb|init_size|
            and each individual has a genetic traits randomly sampled from a uniform distribution 
            in the range $[-1, 1]$ for each trait
            \item The maximum size of population is defined by the \verb|max_size| parameter and 
            when the population size reaches the maximum size, the individuals with the lowest fitness
            values are removed from the population until the population size is equal to \verb|max_size|
        \end{itemize}
    \item[\textbf{3.}] Environmennt:
        \begin{itemize}
            \item Environment is not implemented explicitly, but is realised by the existence of the
            optimal genotype (genetic traits) represented by numpy array of shape \verb|(n_traits,)| filled
            with zeros at the beginning of the simulation 
            \item During the simulation the optimal genotype is changed from generation to generation 
            according to the formula: $\alpha(t)=\alpha(t-1)+c$ where $\alpha(t)$ is the optimal genotype
            at generation $t$ and $c$ is a random numbers sampled from a uniform distribution in the range
            $[0, \text{warm \textunderscore rate}]$ where \verb|warm_rate| is a parameter of the simulation. The optimal 
            genotype is changed in this way to simulate global warming.
            \item At each generation there is a $0.5\%$ chance that the optimal genotype will be random 
            sampled from a uniform distribution in the range $[-1, 1]$ for each trait. This is done to
            simulate the meteor impact.
        \end{itemize}
    \item[\textbf{4.}] Fitness:
        \begin{itemize}
            \item Fitness of each individual is calculated according to the formula:
            $$
            \phi_{\alpha}(o)=\exp \left(-\frac{\|o-\alpha\|}{2 \sigma^{2}}\right)
            $$
            where $\alpha$ is the optimal genotype, $o$ is the genotype of the individual and $\sigma$ is
            a parameter that controls the natural selection mechanism. For $\sigma \rightarrow \infty$ the
            selection mechanism fades away.
            \item The fitness function is implemented as a vectorized function and it is applied to the 
            entire population at once. This is much more efficient computationally.
            \item The value of the fitness is between the range $[0, 1]$ and provides probability of 
            survival of the individual in the next generation explicitly.
        \end{itemize}
    \item[\textbf{5.}] Mutation:
        \begin{itemize}
            \item Mutation is implemented using a custom function that mutates individuals' traits with a probability \verb|p_mut|
            \item During the mutation process, for each individual, a random number is generated from a uniform distribution. If this random number is less than \verb|p_mut|, one of the individual's traits is mutated
            \item The index of the trait to be mutated is chosen randomly, and the mutation value is sampled from a normal distribution with mean 0 and standard deviation \verb|mut_std|
            \item The mutation value is then added to the selected trait, resulting in a new trait value for the individual
            \item This process enables the exploration of the search space and introduces genetic diversity into the population
        \end{itemize}

    \item[\textbf{6.}] Parent selection:
        \begin{itemize}
            \item Parent selection is performed using a custom function that filters individuals based on their fitness values and a competition rate \verb|comp_rate|
            \item For each individual, a random number is generated from a uniform distribution. If this random number is less than the product of $(1 - \verb|comp_rate|)$ and the individual's fitness value, the individual is selected as a parent
            \item This selection method favors individuals with higher fitness values, but it also allows individuals with lower fitness values to have a chance of being selected as parents, promoting diversity in the population
            \item The selected parents are then used to create offspring in the reproduction process
        \end{itemize}
    \item[\textbf{7.}] Reproduction:
        \begin{itemize}
            \item The parents are randomly paired up to create offspring; each pair of parents produces one child
            \item For each pair of parents, the child is created by taking the element-wise average of the parent's genetic traits with a random scaling factor between 2 and 4
        \end{itemize}

\end{itemize}

\section{Results}

\begin{itemize}
    \item \textbf{Optimal mutation probability and effect for adaptation:} \\
    For the given \verb|_CONFIG| parameters:
        \begin{itemize}
            \item \texttt{init\_size}: 100
            \item \texttt{max\_size}: 1000
            \item \texttt{n\_generations}: 50
            \item \texttt{n\_traits}: 3
            \item \texttt{trait\_min}: -1
            \item \texttt{trait\_max}: 1
            \item \texttt{init\_opt\_gen}: \texttt{np.zeros(3)}
            \item \texttt{sel\_std}: 0.95
            \item \texttt{warm\_rate}: 0.01
            \item \texttt{competition\_rate}: 0.1
            \item \texttt{seed}: 2137
        \end{itemize}

    I run the simulation 5 times for \verb|p_mut| and \verb|mut_std|$\in$ \verb|numpy.linspace(0.1, 0.9, 10)|
    and calculated the average survival rate for each combination of \verb|p_mut| and \verb|mut_std| by summing
    the generations where there was at least one individual and dividing it by the number of simulation runs.
    
    
    {\centering
    This is the result:
    \par}

    \begin{figure}[h] % [h] for placement here, use [t] for top or [b] for bottom
        \centering % Center the image
        \includegraphics[width=0.5\textwidth]{mut.png} % Adjust the width, in this case, it's set to 50% of the text width
        \caption{Optimal \texttt{p\_mut} and \texttt{mut\_std}} % Caption for the image
        \label{fig:mut} % Label for referencing the image
    \end{figure}

    \newpage

    \item \textbf{Visualization of evolution in time:} \\
    For the given \verb|_CONFIG| parameters as befor but with \verb|p_mut| = 0.5 and \verb|mut_std| = 0.4
    I run the simulation for 26 generations.
    
    {\centering
    This is the result:
    \par}

    \begin{figure}[h] % [h] for placement here, use [t] for top or [b] for bottom
        \centering % Center the image
        \includegraphics[width=0.5\textwidth]{pca.png} % Adjust the width, in this case, it's set to 50% of the text width
        \caption{Evolution of population using PCA for 2 components} % Caption for the image
        \label{fig:pca} % Label for referencing the image
    \end{figure}

    \begin{figure}[h] % [h] for placement here, use [t] for top or [b] for bottom
        \centering % Center the image
        \includegraphics[width=0.5\textwidth]{size.png} % Adjust the width, in this case, it's set to 50% of the text width
        \caption{Population size and average fitness over time} % Caption for the image
        \label{fig:size} % Label for referencing the image
    \end{figure}

\end{itemize}

\newpage

\section{Conclusion}

As seen in results section, my model is very unstable for a bigger number of generations. It may be due to 
the fact that the after a certain number of generations the optimal traits can go out of the range
of the traits of population. Also events like meteor impact can cause the extinction of the population sooner 
or later. I think that the model can be improved by adding some kind of a mechanism that prevents the optimal 
traits from going out of the range of the traits of the population.

\end{document}
